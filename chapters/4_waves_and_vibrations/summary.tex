\subsection{Resumen}

\begin{tcolorbox}[title=Movimiento Ondulatorio]
  \textbf{Definición:} Un movimiento ondulatorio es un movimiento que se caracteriza por la transferencia de energía a través del espacio sin el acompañamiento de transferencia de materia. Para que exista una onda, se debe aportar energía al medio.
  Las ondas mecánicas pueden ser:
  \begin{itemize}
    \item \textbf{Ondas trasversales}: el movimiento de las partículas es perpendicular a la dirección de propagación del pulso.
    \item \textbf{Ondas longitudinales}: el movimiento de las partículas es paralelo a la dirección de propagación del pulso.
  \end{itemize}
\end{tcolorbox}

\begin{tcolorbox}[title=Ecuación de onda]
  Es la ecuación que describe el movimiento de una onda en un medio. La ecuación de onda es una función de \textbf{dos variables}: la \textbf{posición} y el \textbf{tiempo}. Su valor de salida es la posición vertical de las partículas del medio.
  \[
    y(x, t) = A \sin\left( kx - \omega t \right)
  \]
  donde:
  \begin{itemize}
    \item \(A\) es la amplitud,
    \item \(\omega\) la frecuencia angular,
    \item \(k\) el número de onda.
  \end{itemize}
\end{tcolorbox}

\begin{tcolorbox}[title=Elementos característicos]
  La onda mecánica se caracteriza por:
  \begin{itemize}
    \item Amplitud \(A\): valor máximo respecto de la posición de equilibrio.
    \item Frecuencia angular: \(\omega = 2\pi f\).
    \item Número de onda: \(k = \frac{2\pi}{\lambda}\).
    \item Longitud de onda: \(\lambda = \frac{2\pi}{k} = vT\).
    \item Velocidad de propagación: \(v = \frac{\lambda}{T}\).
    \item Fase \(\vartheta\): es el argumento del seno o coseno de la función de onda.
  \end{itemize}
\end{tcolorbox}

\begin{tcolorbox}[title=Diferencia de fase y distancia recorrida]
  La diferencia de fase se puede calcular entre dos ondas distintas o entre dos puntos de una misma onda.
  \begin{itemize}
    \item \textbf{Ondas distintas:} \(\delta = \vartheta_2 - \vartheta_1\) (diferencia de fases \(\vartheta\) entre ondas).
    \item \textbf{Puntos de la misma onda:} \(\delta = k(x_2 - x_1) = \frac{2\pi}{\lambda} (x_2 - x_1)\).
  \end{itemize}

  La distancia que viaja una onda en un tiempo \(t\) es:
  \[
    d = v t = \frac{\lambda}{T} t
  \]
\end{tcolorbox}

\begin{tcolorbox}[title=Energía y potencia de una onda]
  La energía media de una onda es:
  \[
    E = \frac{1}{2} \mu A^2 \omega^2
  \]
  donde \(\mu\) es la densidad lineal del medio (en kg/m).

  La potencia media de una onda es:
  \[
    P = \frac{1}{2} \mu A^2 \omega^2 v
  \]
\end{tcolorbox}

\begin{tcolorbox}[title=Intensidad de una onda]
  La intensidad de una onda es:
  \[
    I = \frac{P}{S}
  \]
  Que para una una onda que se propaga radialmente como el sonido es:
  \[
    I = \frac{P}{4\pi r^2}
  \]
  ya que la superficie de una esfera es \(S=4\pi r^2\).
\end{tcolorbox}