Un \textbf{capacitor}, también conocido como \textbf{condensador}, es un componente eléctrico pasivo que tiene la capacidad de \textbf{almacenar energía en forma de un campo eléctrico}. Está compuesto por dos conductores (llamados placas) separados por un material dieléctrico, que actúa como aislante.

Cuando se aplica una diferencia de potencial (voltaje) entre las placas, una de ellas acumula carga positiva y la otra carga negativa, generando así un campo eléctrico entre ellas. La capacidad del capacitor para almacenar carga depende de sus características físicas y del dieléctrico utilizado.

La \textbf{capacitancia} es una medida de esa capacidad de almacenamiento de carga eléctrica. Se define como:

\begin{equation}
    C = \frac{Q}{V}
    \label{eq:capacitance}    
\end{equation}

donde:

\begin{itemize}
    \item \( C \) es la \textbf{capacitancia} del capacitor, medida en faradios (\(\si{\farad}\)) (siempre es una cantidad positiva).
    \item \( Q \) es la \textbf{carga eléctrica} almacenada en el capacitor, medida en coulombs (\(\si{\coulomb}\)).
    \item \( V \) es la \textbf{diferencia de potencial} entre las placas del capacitor, medida en voltios (\(\si{\volt}\)).
\end{itemize}

Un faradio es una unidad muy grande, por lo que en la práctica suelen utilizarse submúltiplos como el microfaradio (\(\mu\si{farad}\)), nanofaradio (n\(\si{\farad}\)) o picofaradio (p\(\si{\farad}\)).

La capacitancia depende de factores. Si se tienen placas paralelas de depende de factores como el área de las placas (\(A\)), la distancia entre ellas (\(d\)) y la permitividad del dieléctrico (\( \varepsilon \)) según la siguiente fórmula:

\[
C = \varepsilon \frac{A}{d}
\]

Este comportamiento hace que los capacitores sean ampliamente utilizados en circuitos electrónicos para funciones como almacenamiento de energía, filtrado, acoplamiento y desacoplamiento de señales, entre otras.

\subsubsection{Geometría de los conductores}

Cuando la \hl{geometría de los conductores} es distinta a la de un capacitor de placas planas y paralelas, la expresión de la capacitancia cambia, aunque el principio físico fundamental sigue siendo el mismo: almacenar energía en forma de campo eléctrico entre conductores separados por un dieléctrico.

En general, la \hl{capacitancia depende de la geometría de los conductores y del medio dieléctrico entre ellos}. A continuación, se presentan algunos casos comunes:

\paragraph{1. Capacitor esférico}

Consiste en dos esferas concéntricas de radios \( R_1 \) (interna) y \( R_2 \) (externa). Su capacitancia es:

\[
C = 4\pi \varepsilon_0 \varepsilon_r \frac{R_1 R_2}{R_2 - R_1}
\]

\paragraph{2. Capacitor cilíndrico}

Está formado por dos cilindros coaxiales, uno de radio interno \( a \) y otro de radio externo \( b \), y de longitud \( L \) (suponiendo \( L \gg b \)). La capacitancia es:

\[
C = \frac{2\pi \varepsilon_0 \varepsilon_r L}{\ln(b/a)}
\]

\paragraph{3. Geometrías irregulares o generales}

En geometrías más complejas, la capacitancia no puede obtenerse de forma analítica sencilla. En estos casos se recurre a:

\begin{itemize}
    \item Métodos numéricos
    \item Aproximaciones analíticas
    \item Medición experimental
\end{itemize}

Cuando se cambia la geometría, ya no es válida la fórmula simple \( C = \varepsilon \frac{A}{d} \). Es necesario \textbf{considerar la distribución del campo eléctrico} que surge de la nueva disposición geométrica, y calcular la capacitancia a partir de las definiciones fundamentales, como:

\[
C = \frac{Q}{V}
\]

donde \( V \) ahora debe calcularse usando la ley de Gauss o integrando el campo eléctrico apropiado para la geometría dada.