\section{Flujo eléctrico}
\label{sec:flujo_electrico}

El flujo eléctrico es una medida de la cantidad de campo eléctrico que atraviesa una superficie dada. 

\begin{definition}[Flujo eléctrico]
  El flujo eléctrico \(\Phi_E\) se define como
  \[
  \Phi_E = E \cdot A \cdot \cos(\theta)  \quad \left[\frac{\si{\newton \meter \squared}}{\si{\coulomb}}\right]
  \]
  donde \(E\) es la magnitud del campo eléctrico, \(A\) el área de la superficie considerada y \(\theta\) el ángulo entre la dirección del campo eléctrico y la normal de la superficie.
\end{definition}

\begin{figure}[ht]
  \centering
  \begin{subfigure}[b]{0.37\textwidth}
    \centering
    \includegraphics[width=\textwidth]{electric_flux_a.pdf}
    \caption{flujo sobre un área perpendicular.}
  \end{subfigure}
  \hspace{12pt}
  \begin{subfigure}[b]{0.37\textwidth}
    \centering
    \includegraphics[width=\textwidth]{electric_flux_b.pdf}
    \caption{flujo sobre un área inclinada.}
  \end{subfigure}
  \caption{Flujo eléctrico}
\end{figure}

Otra forma más general de escribir el flujo eléctrico es usando el producto escalar. Para poder usar el producto escalar, se necesitan dos vectores, y en este caso solo se tiene el vector de campo eléctrico \(\vec{E}\). Entonces se define un vector normal a la superficie con igual magnitud al área de la superficie. En definitiva, el producto escalar entre el vector campo eléctrico \(\vec{E}\) y el vector normal al área \(\vec{A}\) es:
\begin{equation}
    \Phi_E = \vec{E} \cdot \vec{A} = E A \cos \theta
    \label{eq:flujo_electrico}
\end{equation}

Hay que tener muy presente que \(\Phi_E\) es un valor escalar, no un vector. Además, otro factor a considerar es que la ecuación \eqref{eq:flujo_electrico} solo se puede utilizar si el campo \(\vec{E}\) es constante. Sin embargo, en general, el campo \(\vec{E}\) no suele ser constante, por tanto una forma de aplicar la expresión \eqref{eq:flujo_electrico} es sumar todas las pequeñas contribuciones de flujo eléctrico en pequeñas porciones de área \(dA\), resultando en una integral,
\begin{equation*}
  \Phi_E = \int_{S} \vec{E} \cdot d\vec{A}
\end{equation*}

\subsection{Ley de Gauss}
\label{sec:ley_de_gauss}

El flujo eléctrico es importante en el estudio del electromagnetismo porque está relacionado con la ley de Gauss, que establece que el flujo eléctrico a través de una superficie cerrada (con frecuencia llamada \textit{superficie gaussiana}) es proporcional a la carga eléctrica total encerrada dentro de esa superficie. Suponga una carga puntual positiva \(q\) ubicada en el centro de una esfera de radio \(r\) como se observa en la figura \ref{fig:superficie_gaussiana}.
\begin{marginfigure}
  \includegraphics[width=\marginparwidth]{gauss_surface.pdf}
  \caption{Superficie gaussiana esférica de radio \(r\) que rodea una carga puntual \(q\).}
  \label{fig:superficie_gaussiana}
\end{marginfigure}

De la ecuación de campo eléctrico, se sabe que la magnitud del campo eléctrico sobre todos los puntos de la superficie (\(S\)) de la esfera es \(\vec{E} = k q/r^2\).

Las líneas de campo están dirigidas radialmente hacia afuera y por tanto son perpendiculares a la superficie en todos sus puntos. Es decir, en cada punto de la superficie, \(\vec{E}\) es paralelo al vector \(d\vec{A}\) que representa un elemento de área muy pequeño que rodea al punto en la superficie. Por lo tanto, el flujo neto a través de la superficie gaussiana es igual a
\begin{align*}
  \Phi_E =& \oint_S \vec{E} \cdot d\vec{A} \\
          =& \oint_S E \cos(0) ~ dA \\
          =& \oint_S E ~ dA \\
  \Phi_E =& E ~ \oint_S dA
\end{align*}
aquí se tiene que \(E=kq/r^2\) donde \(r\) es conocido y vale el radio de la esfera, \(q\) es el valor de la carga en el centro de la esfera y \(k = 1/(4\pi\varepsilon_0)\). Por otro lado, la integral resulta ser la superficie de una esfera cuyo valor es \(4\pi r^2\). Entonces,

\[
\Phi_E = E ~ \oint_S dA = \frac{q}{4\pi \varepsilon_0 r^2} \cdot 4\pi r^2 = \boxed{\frac{q}{\varepsilon_0}} 
\]

Este resultado dice muchas cosas:
\begin{itemize}
  \item \textit{Primero}: el flujo eléctrico \(\Phi_E\) no depende del radio de la superficie esférica.
  \item \textit{Segundo}: el flujo es directamente proporcional a la carga encerrada en el interior de la superficie gaussiana.
  \item \textit{Tercero}: el flujo es inversamente proporcional al valor de \(\varepsilon\), en el ejemplo se trabajó con vacío, pero puede ser cualquier material no conductor con otro valor de \(\varepsilon\).
\end{itemize}

En base a esto se puede sacar una conclusión, suponga que la superficie no es esférica ¿Cómo será el flujo? El flujo, sorprendentemente, será el mismo que el de la esfera, pero ¿Por qué? Piense en la definición de flujo, sea cualquier superficie que rodea la esfera, el flujo será igual a el \(\vec{E}\cdot d\vec{A}\). Como se vio anteriormente esto representa la cantidad de líneas de campo que pasan por una superficie determinada. Como la superficie encierra la misma carga entonces saldrán la misma cantidad de líneas de campo por la superficie.

En este punto tal vez se pregunte ¿Cómo puede ser que no dependa del radio de la esfera, o mejor, del tamaño de la superficie arbitraria cerrada? Es sencillo, si el tamaño de la superficie cerrada aumenta, entonces la distancia a la carga encerrada también aumentará, esto significa que la intensidad del campo disminuirá, entonces pasarán menos líneas de campo por cada \(dA\), pero como la superficie total a aumentado, entonces compensará la pérdida de intensidad del campo. 

Como conclusión, el flujo neto a través de \textit{cualquier} superficie cerrada que rodea a una carga puntual \(q\) está dado por \(q/\varepsilon_0\) y es independiente de la forma de la superficie. Entonces, a modo de ejemplo, si supone múltiples superficies, llámese cada una \(S1\), \(S2\) y \(S3\) y estas superficies rodean a la carga \(q\) (ver figura \ref{fig:superficie_gaussiana_arbitraria} del lado izquierdo), entonces todas estas superficies tienen el mismo flujo eléctrico \(\Phi_E\), ya que todas encierran la misma carga \(q\). 

\begin{figure}[ht]
  \centering
  \includegraphics[width=0.7\textwidth]{gauss_law_cases.pdf}
  \caption{Distintas superficies gaussianas de forma arbitraria en presencia de una carga puntual \(q\).}
  \label{fig:superficie_gaussiana_arbitraria}
\end{figure}

Por otro lado, si la carga \(q\) está fuera de la superficie gaussiana entonces la misma cantidad de líneas de campo que entran por un lado de la superficie, salen por el otro lado. Por lo tanto, el flujo neto a través de la superficie es cero. Esto se puede ver en la figura \ref{fig:superficie_gaussiana_arbitraria} del lado derecho, donde se observa que el flujo neto es cero.

La forma matemática de ley de Gauss, es una generalización de lo anterior y establece que el flujo neto a través de cualquier superficie cerrada es
\begin{equation}
  \Phi_E = \oint_{S} \vec{E} \cdot d\vec{A} = \frac{q_\text{int}}{\varepsilon_0} \quad \left[\frac{\si{\newton \meter \squared}}{\si{\coulomb}}\right]
\end{equation}
donde \( \Phi_E \) es el flujo de campo eléctrico a través de una superficie cerrada, \(E\) es el campo eléctrico en cada punto de esa superficie, \(dA\) es un vector diferencial de área que apunta hacia afuera de la superficie, \(q_\text{int}\) es la carga total encerrada dentro de la superficie y \(\varepsilon_0\) es la permitividad del vacío, una constante del medio.

\begin{tcolorbox}[mydanger]
  Atención, \(\vec{E}\) representa el campo eléctrico total que incluye contribuciones de ambas cargas, tanto del interior, como del exterior de la superficie.    
\end{tcolorbox}

La ley de Gauss establece un resultado para superficies cerradas. Conceptualmente, establece que si hay carga en el interior de una superficie cerrada el campo eléctrico ``sale'' (o ``entra'') por esa superficie, generando un flujo distinto de cero. Si no hay carga neta dentro, el flujo eléctrico total es cero, aunque el campo pueda no ser nulo en todos los puntos. Este resultado se aplica indistintamente de la forma de la superficie, la distribución del campo eléctrico y el tipo de carga encerrada.

Esta ley es útil porque, en situaciones con alta simetría (esférica, cilíndrica, planar), permite calcular el campo eléctrico sin integrar la ley de Coulomb. Por ejemplo, para una esfera cargada uniformemente, para un hilo infinitamente largo con distribución de carga lineal y uniforme o para un plano infinito cargado es posible aplicar la ley de Gauss y obtener el campo eléctrico. En cada caso es importante la correcta elección de una superficie gaussiana conveniente, ya que la utilidad es simplificar un problema.

\begin{example}
Suponga una carga distribuida uniformemente en una esfera. Si elige una superficie esférica de radio mayor al de la esfera, por simetría:
\[
  \vec{E} = \text{constante} \quad \text{y} \quad \vec{E} \parallel d\vec{A}
\]
Entonces la integral se simplifica:
\[
  \oint \vec{E} \cdot d\vec{A} = E \oint dA = E(4\pi r^2)
\]
Y por la ley de Gauss:
\[
  E(4\pi r^2) = \frac{Q}{\varepsilon_0} \quad \Rightarrow \quad E = \frac{1}{4\pi\varepsilon_0} \frac{Q}{r^2}
\]
¡Y así se recupera la fórmula del campo eléctrico de una carga puntual!
\end{example}
