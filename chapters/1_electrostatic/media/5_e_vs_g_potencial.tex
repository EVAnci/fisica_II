\begin{figure}[!ht]
  \centering
  \begin{subfigure}[b]{0.45\textwidth}
    \centering
    \begin{tikzpicture}[>=stealth]
      \coordinate (A) at (0,1);
      \coordinate (B) at (0,-1);
      \begin{scope}[very thick,decoration={
          markings,
          mark=at position 0.6 with {\arrow{>}}
        }] 
        \foreach \x in {-1,-.5,...,1} {
          \draw[orange, thick,postaction=decorate] (\x,1.5) -- (\x,-1.5);
        }
      \end{scope}
      \shade[ball color=orange, opacity=0.3, text opacity=1] (A) circle (.2) node[left=1mm] {\scriptsize{A}};
      \shade[ball color=orange] (B) circle (.2) node[left=1mm] {\scriptsize{B}};
      \draw[thin] ($(A)+(0.4,0)$) -- ($(A)+(0.9,0)$);
      \draw[thin] ($(B)+(0.4,0)$) -- ($(B)+(0.9,0)$);
      \draw[thin,<->] ($(A)+(.75,0)$) -- node[fill=white,inner sep=1pt] {$d$} ($(B)+(.75,0)$);
      \node at (-1.3,-1) {$\vec{E}$};

      \node (cartel) [
        font=\footnotesize,
        text width=4.5cm,
        align=left,
        fill=yellow!40!black!14,
        draw=yellow!40!black!46,
        thick,
        rounded corners=2pt,
        inner sep=2mm
        ]
        at ($(A)+(0,1.7)$) {Cuando una carga de prueba positiva se mueve del punto A al punto B, la energía potencial eléctrica del sistema carga-campo disminuye.};
    \end{tikzpicture}
  \end{subfigure}
  \begin{subfigure}[b]{0.45\textwidth}
    \centering
    \begin{tikzpicture}[>=stealth]
      \coordinate (A) at (0,1);
      \coordinate (B) at (0,-1);
      \begin{scope}[very thick,decoration={
          markings,
          mark=at position 0.6 with {\arrow{>}}
        }] 
        \foreach \x in {-1,-.5,...,1} {
          \draw[blue!50, thick,postaction=decorate] (\x,1.5) -- (\x,-1.5);
        }
      \end{scope}
      \shade[ball color=blue!50, opacity=0.3, text opacity=1] (A) circle (.2) node[left=1mm] {\scriptsize{A}};
      \shade[ball color=blue!50] (B) circle (.2) node[left=1mm] {\scriptsize{B}};
      \draw[thin] ($(A)+(0.4,0)$) -- ($(A)+(0.9,0)$);
      \draw[thin] ($(B)+(0.4,0)$) -- ($(B)+(0.9,0)$);
      \draw[thin,<->] ($(A)+(.75,0)$) -- node[fill=white,inner sep=1pt] {$d$} ($(B)+(.75,0)$);
      \node at (-1.3,-1) {$\vec{g}$};

      \node (cartel) [
        font=\footnotesize,
        text width=4.5cm,
        align=left,
        fill=yellow!40!black!14,
        draw=yellow!40!black!46,
        thick,
        rounded corners=2pt,
        inner sep=2mm
        ]
        at ($(A)+(0,1.7)$) {Cuando un objeto con masa se mueve del punto A al punto B, la energía potencial gravitacional del sistema objeto-campo disminuye.};
    \end{tikzpicture}
  \end{subfigure}
  \caption{Comparación de la energía potencial eléctrica y gravitatoria.}
  \label{fig_potential_uniform}
\end{figure}
