\begin{tikzpicture}[>=stealth]
  \begin{axis}[
      axis lines = middle, 
      xlabel={$r$},
      ylabel={$U$},
      ymin=-0.5, ymax=6.5,
      xmin=-0.5, xmax=6.5,
      xtick=\empty,
      ytick=\empty,
      domain=0.17:5.5, 
      samples=100,
      grid=none,
      width=\marginparwidth,
      height=\marginparwidth,
      axis line style={gray}
    ]
    \addplot[teal, very thick] {1/x};
  \end{axis}
  \shade[ball color=orange] (1,2) circle (.15) node[above=1mm] {\scriptsize$q$};
  \shade[ball color=orange] (2,2) circle (.15) node[above=1mm] {\scriptsize$q_0$};
  \draw[thin] (1,1.8) -- (1,1.6);
  \draw[thin] (2,1.8) -- (2,1.6);
  \draw[thin,<->] (1,1.7) -- node[below] {$r$} (2,1.7);
  \shade[ball color=blue!40] (1,2.7) circle (.15) node[above=1mm] {\scriptsize$q$};
  \shade[ball color=blue!40] (2,2.7) circle (.15) node[above=1mm] {\scriptsize$q_0$};
\end{tikzpicture}
\vspace{1cm}
\begin{tikzpicture}[>=stealth]
  \begin{axis}[
      axis lines = middle, 
      xlabel={$r$},
      ymin=-6.5, ymax=0.5,
      xmin=-0.5, xmax=6.5,
      xtick=\empty,
      ytick=\empty,
      domain=0.17:5.5, 
      samples=100,
      grid=none,
      width=\marginparwidth,
      height=\marginparwidth,
      axis line style={gray}
    ]
    \addplot[teal, very thick] {-1/x};
  \end{axis}
  \shade[ball color=blue!40] (1,1) circle (.15) node[above=1mm] {\scriptsize$q$};
  \shade[ball color=orange] (2,1) circle (.15) node[above=1mm] {\scriptsize$q_0$};
  \draw[thin] (1,0.8) -- (1,0.6);
  \draw[thin] (2,0.8) -- (2,0.6);
  \draw[thin,<->] (1,0.7) -- node[below] {$r$} (2,0.7);
  \shade[ball color=orange] (1,1.7) circle (.15) node[above=1mm] {\scriptsize$q$};
  \shade[ball color=blue!40] (2,1.7) circle (.15) node[above=1mm] {\scriptsize$q_0$};
  \node (cartel) [
    text width=3cm,
    align=left,
    fill=yellow!40!black!14,
    draw=yellow!40!black!46,
    thick,
    rounded corners=2pt,
    inner sep=2mm
    ]
    at (1.35,-2) {\footnotesize{Cuando las cargas son iguales, el potencial es mayor si la distancia es menor. Lo contrario ocurre cuando son distintas.}};
\end{tikzpicture}
