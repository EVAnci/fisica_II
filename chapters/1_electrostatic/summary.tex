\section{Resumen}

\begin{tcolorbox}[resumen,title=Ley de Coulomb]
  Fuerza entre dos cargas puntuales \(q_1\) y \(q_2\):
  \[
    \vec{F} = k \frac{q_1 q_2}{r^2} \hat{r}
  \]
  \begin{itemize}
    \item \(k = \frac{1}{4\pi\varepsilon_0}\approx \qty{9e9}{\newton\meter\squared\per\coulomb\squared}\) es la constante de Coulomb.
    \item \(r\) es la distancia entre las cargas.
    \item \(\hat{r}\) es el vector unitario radial.
  \end{itemize}
\end{tcolorbox}

\begin{tcolorbox}[resumen,title=Campo Eléctrico]
  Campo eléctrico \(\vec{E}\) en un punto debido a una carga \(Q\):
  \[
    \vec{E} = \frac{\vec{F}}{q_0} = k \frac{Q}{r^2} \hat{r}
  \]
  donde \(\vec{F}\) es la fuerza sobre una carga de prueba \(q_0\).

  \tcbline

  Campo eléctrico a partir del potencial eléctrico \(V\):
  \[
    \vec{E} = -\nabla V
  \]
  donde \(\nabla\) es el operador nabla (gradiente).
 
  \tcbline

  Campo eléctrico debido a una distribución de carga continua:
  \[
    \vec{E} = k \int \frac{dq}{r^2} \hat{r}
  \]
  donde \(dq\) es un elemento diferencial de carga. La integración se realiza sobre la distribución de carga (\(\lambda\), \(\sigma\) o \(\rho\)).
\end{tcolorbox}

\begin{tcolorbox}[resumen,title=Flujo Eléctrico]
  Flujo eléctrico \(\Phi_E\) a través de una superficie \(S\):
  \[
    \Phi_E = \int_S \vec{E} \cdot d\vec{A}
  \]
  donde \(d\vec{A}\) es el vector área (normal a la superficie). El flujo representa el ``número de líneas de campo'' que atraviesan \(S\).
\end{tcolorbox}

\begin{tcolorbox}[resumen,title=Ley de Gauss]
  Ley de Gauss para un campo eléctrico \(\vec{E}\):
  \[
    \oint_S \vec{E} \cdot d\vec{A} = \frac{Q_{\text{int}}}{\varepsilon_0}
  \]
  donde \(Q_{\text{int}}\) es la carga encerrada por la superficie Gaussiana \(S\). En esta ecuación se relaciona el flujo eléctrico con la carga total encerrada.
\end{tcolorbox}

\begin{tcolorbox}[resumen,title=Trabajo Eléctrico]
  Trabajo \(W_\text{int}\) realizado por el campo eléctrico al mover una carga \(q\) desde un punto \(A\) a un punto \(B\):
  \[
    W_{AB} = -\Delta U = q \Delta V = \int_{A}^{B} q\vec{E} \cdot d\vec{s}
  \]
  donde \(\Delta U\) es el cambio en energía potencial.
\end{tcolorbox}

\begin{tcolorbox}[resumen,title=Energía Potencial]
  Energía potencial eléctrica \(U\) de una carga \(q\) en un campo eléctrico \(\vec{E}\):
  \[
    U = qV = k\frac{qQ}{r}
  \]
  
  \tcbline 

  La energía eléctrica almacenada por un capacitor:
  \[
    U = \frac{1}{2} CV^2 = \frac{1}{2} QV = \frac{1}{2} \frac{Q^2}{C}
  \]
\end{tcolorbox}
  
\begin{tcolorbox}[resumen,title=Potencial Eléctrico]
  Definición de potencial en un punto \(p\) debido a una carga puntual \(Q\):
  \[
    V = -\int_{\infty}^{p} \vec{E} \cdot d\vec{s} = k \frac{Q}{p}
  \]
  
  \tcbline 

  Diferencia de potencial eléctrico \(\Delta V\) entre dos puntos \(A\) y \(B\):
  \[
    \Delta V = V_B - V_A = -\int_{A}^{B} \vec{E} \cdot d\vec{s}
  \]
  
  \tcbline 

  Potencial para un campo uniforme y constante:
  \[
    V = -Ed
  \]
  Recuerde que el potencial \(V\) es un valor escalar.
\end{tcolorbox}

\begin{tcolorbox}[resumen,title=Capacitancia]
  Capacitancia \(C\) de un capacitor: 
  \[
    C = \frac{Q}{V} ~ [\unit{\farad}]
  \]
  donde \(Q\) es la carga almacenada, \(V\) es el potencial.
  \tcbline
  Capacitancia de un condensador con dieléctrico:
  \[
    C = \mathcal{K}C_0
  \]
  donde \(\mathcal{K}\) es el factor de multiplicación o permitividad relativa y \(C_0\) es la capacitancia con \(\varepsilon_0\).
  \tcbline
  Capacitancia equivalente:
  \begin{itemize}
    \item En serie:
      \[
        \frac{1}{C_{\text{eq}}} = \sum_{i=1}^{n} \frac{1}{C_i}
      \]
    \item En paralelo:
      \[
        C_{\text{eq}} = \sum_{i=1}^{n} C_i
      \]
  \end{itemize}
\end{tcolorbox}
