\subsection{Resumen}

\begin{tcolorbox}[title=Ley de Coulomb]
  Fuerza entre dos cargas puntuales \(q_1\) y \(q_2\):
  \[
    \vec{F} = k \frac{q_1 q_2}{r^2} \hat{r}
  \]
  \begin{itemize}
    \item \(k = \frac{1}{4\pi\varepsilon_0}\) es la constante de Coulomb.
    \item \(r\) es la distancia entre las cargas.
    \item \(\hat{r}\) es el vector unitario radial.
  \end{itemize}
\end{tcolorbox}

\begin{tcolorbox}[title=Flujo Eléctrico]
  Flujo eléctrico \(\Phi_E\) a través de una superficie \(S\):
  \[
    \Phi_E = \int_S \vec{E} \cdot d\vec{A}
  \]
  \begin{itemize}
    \item \(d\vec{A}\) es el vector área (normal a la superficie).
    \item[\textbf{Idea:}] Representa el ``número de líneas de campo'' que atraviesan \(S\).
  \end{itemize}  
\end{tcolorbox}

\begin{tcolorbox}[title=Ley de Gauss]
  Ley de Gauss para un campo eléctrico \(\vec{E}\):
  \[
    \oint_S \vec{E} \cdot d\vec{A} = \frac{Q_{\text{int}}}{\varepsilon_0}
  \]
  \begin{itemize}
    \item \(Q_{\text{int}}\) es la carga encerrada por la superficie Gaussiana \(S\).
    \item[\textbf{Idea:}] Relaciona el flujo eléctrico con la carga total encerrada.
  \end{itemize}  
\end{tcolorbox}

\begin{tcolorbox}[title=Campo Eléctrico]
  Campo eléctrico \(\vec{E}\) en un punto debido a una carga \(Q\):
  \[
    \vec{E} = \frac{\vec{F}}{q_0} = k \frac{Q}{r^2} \hat{r}
  \]
  \begin{itemize}
    \item \(\vec{F}\) es la fuerza sobre una carga de prueba \(q_0\).
    \item Dirección radial para cargas puntuales.
  \end{itemize}

  Campo eléctrico a partir del potencial eléctrico \(V\):
  \[
    \vec{E} = -\nabla V
  \]
  \begin{itemize}
    \item \(\nabla\) es el operador nabla (gradiente).
  \end{itemize}
  Campo eléctrico debido a una distribución de carga continua:
  \[
    \vec{E} = k \int \frac{dq}{r^2} \hat{r}
  \]
  \begin{itemize}
    \item \(dq\) es un elemento diferencial de carga.
    \item Integración sobre la distribución de carga (\(\lambda\), \(\sigma\) o \(\rho\)).
    \item \(r\) es la distancia desde el elemento de carga al punto de interés.
  \end{itemize}
\end{tcolorbox}

\begin{tcolorbox}[title=Energía Potencial]
  Energía potencial eléctrica \(U\) de una carga \(q\) en un campo eléctrico \(\vec{E}\):
  \[
    U = qV = k\frac{qQ}{r}
  \]
  \begin{itemize}
    \item \(V\) es el potencial eléctrico en el punto donde se encuentra la carga.
  \end{itemize}
  La energía eléctrica almacenada por un capacitor:
  \[
    U = \frac{1}{2} CV^2 = \frac{1}{2} QV = \frac{1}{2} \frac{Q^2}{C}
  \]
  \begin{itemize}
    \item \(C\) es la capacitancia del capacitor.
    \item \(V\) es el potencial entre las placas del capacitor.
    \item \(U\) es la energía almacenada en el campo eléctrico entre las placas.
  \end{itemize}
\end{tcolorbox}
  
\begin{tcolorbox}[title=Potencial Eléctrico]
  Definición de potencial en un punto \(r\) debido a una carga \(Q\):
  \[
    V = -\int_{\infty}^{r} \vec{E} \cdot d\vec{s} = k \frac{Q}{r}
  \]
  Diferencia de potencial eléctrico \(\Delta V\) entre dos puntos \(A\) y \(B\):
  \[
    \Delta V = V_B - V_A = -\int_{A}^{B} \vec{E} \cdot d\vec{s}
  \]
  Potencial para un campo constante:
  \[
    V = -Ed
  \]
  \textbf{Notas:}
  \begin{itemize}
    \item Energía potencial por unidad de carga.
    \item Medido en voltios (V).
    \item \(V\) es escalar, no vectorial.
  \end{itemize}
\end{tcolorbox}

\begin{tcolorbox}[title=Trabajo Eléctrico]
  Trabajo \(W\) realizado por el campo eléctrico al mover una carga \(q\) desde un punto \(A\) a un punto \(B\):
  \[
    W_{AB} = -\Delta U = q \Delta V = \int_{A}^{B} q\vec{E} \cdot d\vec{s}
  \]
  \begin{itemize}
    \item \(\Delta U\) es el cambio en energía potencial.
    \item \(W\) es positivo si el campo realiza trabajo sobre la carga.
  \end{itemize}
\end{tcolorbox}

\begin{tcolorbox}[title=Capacitancia]
  Capacitancia \(C\) de un condensador:
  \[
    C = \frac{Q}{V}
  \]
  \begin{itemize}
    \item \(Q\) es la carga almacenada, \(V\) es el potencial.
    \item Medido en faradios (F).
  \end{itemize}
  Capacitancia de un condensador con dieléctrico:
  \[
    C = \varepsilon_r C_0
  \]
  \begin{itemize}
    \item \(\varepsilon_r\) es la constante dieléctrica.
    \item \(C_0\) es la capacitancia en el vacío.
    \item Aumenta la capacitancia al introducir un dieléctrico.
  \end{itemize}
  Capacitancia equivalente:
  \begin{itemize}
    \item En serie:
      \[
        \frac{1}{C_{\text{eq}}} = \sum_{i=1}^{n} \frac{1}{C_i}
      \]
    \item En paralelo:
      \[
        C_{\text{eq}} = \sum_{i=1}^{n} C_i
      \]
  \end{itemize}
\end{tcolorbox}

\newpage