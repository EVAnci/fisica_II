\begin{table}[h]
    \centering
    \renewcommand{\arraystretch}{1.5}
    \begin{tabular}{|>{\bfseries}l|l|}
        \hline
        \textbf{Concepto} & \textbf{Definición Matemática y Explicación} \\ \hline
        
        \textbf{Ley de Coulomb} & 
        $\vec{F} = k \frac{q_1 q_2}{r^2} \hat{r}$ \\
        & Fuerza entre dos cargas puntuales ($q_1$, $q_2$): \\
        & - $k = \frac{1}{4\pi\varepsilon_0}$ (Constante de Coulomb) \\
        & - $r$: Distancia entre cargas, $\hat{r}$: Vector unitario radial. \\ \hline
        
        \textbf{Campo Eléctrico} & 
        $\vec{E} = \frac{\vec{F}}{q_0} = k \frac{Q}{r^2} \hat{r}$ \\
        & Fuerza por unidad de carga ($q_0$) en un punto: \\
        & - Dirección: Radial para cargas puntuales. \\ \hline
        
        \textbf{Flujo Eléctrico} & 
        $\Phi_E = \int_S \vec{E} \cdot d\vec{A}$ \\
        & Medida del "número de líneas de campo" que atraviesan \\
        & una superficie $S$: \\
        & - $d\vec{A}$: Vector área (normal a la superficie). \\ \hline
        
        \textbf{Ley de Gauss} & 
        $\oint \vec{E} \cdot d\vec{A} = \frac{Q_{\text{int}}}{\varepsilon_0}$ \\
        & Relación entre flujo eléctrico a través de una superficie \\
        & cerrada y la carga encerrada ($Q_{\text{int}}$). \\ \hline
        
        \textbf{Energía Potencial} & 
        $U = k \frac{q_1 q_2}{r}$ \\
        & Trabajo para reunir cargas desde el infinito: \\
        & - $U > 0$ (repulsión), $U < 0$ (atracción). \\ \hline
        
        \textbf{Trabajo Eléctrico} & 
        $W = -\Delta U = q \Delta V$ \\
        & Trabajo realizado por el campo para mover una carga $q$: \\
        & - Depende de la diferencia de potencial ($\Delta V$). \\ \hline
        
        \textbf{Potencial Eléctrico} & 
        $V = \frac{U}{q_0} = k \frac{Q}{r}$ \\
        & Energía potencial por unidad de carga ($q_0$): \\
        & - Escalar, medido en voltios (V). \\ \hline
    \end{tabular}
    \caption{Resumen de conceptos fundamentales de Electroestática.}
    \label{tab:electrostatica}
\end{table}