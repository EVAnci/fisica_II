La electrostática estudia las cargas eléctricas en \hl{\textbf{reposo}}. La ley de Coulomb establece que \textbf{la fuerza entre dos cargas puntuales} es:

\begin{equation}
    F = k_e \frac{\abs{q_1} \cdot \abs{q_2}}{r^2}
    \label{eq:ley_coulomb}
\end{equation}

donde:

\begin{itemize}
    \item \( k_e \) es la constante de Coulomb: \( 8.9876 \times 10^9 \, \frac{\si{\newton\meter\squared}}{\si{\coulomb\squared}} \).
    \begin{itemize}
        \item \( k_e \) se obtiene de  \( \frac{1}{4\pi\epsilon_0} \) y,
        \item \( \epsilon_0 \) es la permitividad del vacío: \( 8.85 \times 10^{-12} \,\frac{\si{\coulomb\squared}}{\si{\newton\meter\squared}} \)
    \end{itemize}
    \item \( q_1 \) y \( q_2 \) son las cargas.
    \item \( r \) es la distancia entre las cargas.
\end{itemize}

Es muy importante notar que la \textbf{Ley de Coulomb} \eqref{eq:ley_coulomb} se aplica estrictamente a \hl{cargas puntuales}, es decir, cargas que se consideran concentradas en un solo punto sin dimensiones espaciales. En la ecuación anterior \eqref{eq:ley_coulomb}, se obtiene la magnitud de la fuerza eléctrica. La expresión vectorial de la fuerza eléctrica es:

\begin{equation}
    \vec{F}_e = k_e \frac{|q_1 q_2|}{r^2} \hat{r}
    \label{eq:ley_coulomb_vectorial}
\end{equation}

donde:
\begin{itemize}
    \item \( \vec{F}_e \) es el vector fuerza eléctrica entre dos cargas puntuales \( q_1 \) y \( q_2 \)
    \item \( \hat{r} \) es el vector unitario en la dirección que une ambas cargas.
\end{itemize}

Esto es importante tenerlo en cuenta ya que si se quiere saber la fuerza total sobre una carga \( q \) debido a varias cargas, se debe sumar vectorialmente las fuerzas individuales. La fuerza total sobre una carga \( q \) debido a un conjunto de cargas \( Q_i \) es:

\begin{equation}
    \vec{F} = \sum_i k \frac{|q \cdot Q_i|}{r_i^2} \hat{\mathbf{r_i}}
    \label{eq:ley_coulomb_vectorial_suma}
\end{equation}

\subsubsection{Cargas eléctricas}

La \textbf{carga eléctrica} es una propiedad fundamental de la materia que determina la interacción electromagnética entre partículas. Se trata de una magnitud escalar que puede ser de dos tipos: \textbf{positiva} o \textbf{negativa}. Las partículas con carga del mismo signo se repelen, mientras que las de signo opuesto se atraen.

La unidad de carga eléctrica en el \textbf{Sistema Internacional (SI)} es el \textbf{coulomb (\( \si{coulomb} \))}. La carga elemental está representada por la carga del electrón (\( -e = -1.602 \times 10^{-19} \si{coulomb} \)) y la del protón (\( e = +1.602 \times 10^{-19} \si{coulomb} \)).

\begin{center}
    \setlength{\arrayrulewidth}{1pt}  % Grosor de líneas
    \renewcommand{\arraystretch}{1.3} % Espaciado vertical
    \arrayrulecolor{gray} % Color de líneas

    \begin{tabular}{ c c c }
        \hline
        \rowcolor{asparagus!30}
        \textbf{Partícula}  & \textbf{Carga (\si{C})}           & \textbf{Masa (\si{kg})}   \\ \hline
        Electrón (e)        & \(-e = -1.602 \times 10^{-19}\)   & \(9.109 \times 10^{-31}\) \\
        Protón (p)          & \(+e = +1.602 \times 10^{-19}\)   & \(1.672 \times 10^{-27}\) \\
        Neutrón (n)         & \(0\)                             & \(1.675 \times 10^{-27}\) \\ \hline
    \end{tabular}
\end{center}