\subsection{Lineas de campo eléctrico}
\label{sec:lineas_de_campo}

El concepto de líneas de campo eléctrico sirve para visualizar la acción de los campos eléctricos en el espacio. Aunque es una herramienta gráfica y no una entidad física real, permite representar de forma intuitiva la dirección y magnitud del campo eléctrico generado por distribuciones de carga.
\begin{marginfigure}
  \includegraphics[width=\marginparwidth]{field_lines.pdf}
  \caption{Líneas de campo eléctrico que atraviesan dos superficies}
  \label{fig:lineas_de_campo}
\end{marginfigure}

Una línea de campo eléctrico (también llamada línea de fuerza) es una curva imaginaria trazada en el espacio tal que en cada punto de ella, el vector campo eléctrico \(\vec{E}\) es tangente a la curva.

Las líneas de campo eléctrico presentan las siguientes propiedades esenciales: 

\begin{property}[Dirección]
Apuntan en el sentido en que una carga de prueba positiva se movería si se dejara libre en el campo (ya que parten del concepto de carga de prueba). Por lo tanto salen de cargas positivas y entran en cargas negativas. De esta propiedad generalmente las cargas positivas adoptan el nombre de ``generadores'' y las cargas negativas ``sumideros''.
\end{property}

\begin{property}[Disjuntas]
Las líneas nunca se cruzan, si dos líneas se cruzaran, significaría que en ese punto el campo tiene dos direcciones diferentes, lo cual es físicamente imposible.
\end{property}

\begin{property}[Inicio y fin]
Las líneas comienzan y terminan en cargas, en el caso de un campo estático, todas las líneas comienzan en cargas positivas y terminan en cargas negativas. En el infinito, pueden comenzar o terminar si hay una distribución desequilibrada de carga.
\end{property}

Adicionalmente, además de las propiedades, la densidad de líneas representa la magnitud del campo. Cuanto más juntas están las líneas, mayor es la intensidad \(|\vec{E}|\) del campo en esa región (como se muestra en la figura \ref{fig:lineas_de_campo}). En otras palabras el número de líneas de campo que pasan por una superficie perpendicular a dichas líneas es proporcional a la magnitud de campo.

En términos simples este concepto es sencillo ¿Recuerda el ``mapa de flechas''? Las lineas de campo representan exactamente ese mapa de una forma más práctica. Vea, la figura \ref{fig:ej_lineas_de_campo} representa el mismo diagrama de tres cargas pero empleando líneas de campo en vez de calcular el campo en múltiples puntos.

\begin{marginfigure}
  \centering
  \includegraphics[width=\marginparwidth]{field_lines_ex.png}
  \caption{Ejemplo del ``mapa de flechas'' visualizado con líneas de campo}
  \label{fig:ej_lineas_de_campo}
\end{marginfigure}

En este caso puede ver que las líneas de campo salen de cargas positivas y entran en cargas negativas. Además, están más juntas en las proximidades de las cargas, y se van separando al alejarse. La proximidad entre líneas de campo indica que el campo es intenso. Si las líneas están distantes entre sí indica que el campo es menos intenso. Esto se puede verificar con la ecuación del campo eléctrico, ya que el campo disminuye con el cuadrado de la distancia. 
