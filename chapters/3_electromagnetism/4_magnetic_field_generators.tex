\subsection{Fuentes de campos magnéticos}

Anteriormente probablemente haya visto que partículas con masa generan un campo gravitatorio y pueden interactuar con otros campos gravitatorios. También que cargas eléctricas generan un campo eléctrico y pueden interactuar con otros campos eléctricos. Para las cargas en movimiento no es la excepción, tal y como las otras interacciones, las cargas en movimiento interactúan con campos magnéticos y generan campos magnéticos.

Antes de continuar es importante aclarar un detalle importante en la analogía de arriba. Que un objeto con masa genere un campo gravitatorio \textbf{no} significa que los campos gravitatorios interactuen entre sí. Lo mismo para los campos eléctricos y magnéticos. La interacción no es campo-campo, sino más bien \textit{objeto-campo}. En el caso del campo magnético la interacción ocurre entre una carga en movimiento y un campo magnético, o entre una corriente y un campo magnético.

\subsubsection{Campo magnético de una carga en movimiento}

Para poder explicarlo veamos la figura \ref{fig:lineas_de_campo_magnético_de_la_carga_puntual}. En esta figura se muestran las líneas de campo magnético generadas por una carga puntual que se mueve con velocidad constante.

\begin{figure}[ht]
  \centering
  \includegraphics[width=0.3\textwidth]{squematic_charge_magnetic_field.png}
  \caption{Líneas de campo magnético de la carga puntual.}
  \label{fig:lineas_de_campo_magnético_de_la_carga_puntual}
\end{figure}

El campo magnético \(\vec{B}\) generado por la carga puntual \(q\) que se mueve con velocidad constante \(\vec{v}\) en \textbf{un punto del espacio} \(\vec{r}\) está dado por:

\begin{equation}
  \vec{B} = \frac{\mu_0}{4\pi} \frac{q \, \vec{v} \times \hat{r}}{r^2}
  \label{eq:campo_magnético_de_una_carga_puntual}
\end{equation}
donde:
\begin{itemize}
  \item \(\mu_0 = 4 \pi \times 10^{-7} \, \frac{\si{\newton}}{\si{\ampere\squared}}\) es la permeabilidad del vacío,
  \item \(\vec{r}\) es el vector que apunta desde la posición de la carga al punto donde se evalúa el campo,
  \item \(r = |\vec{r}|\) es la magnitud del vector \(\vec{r}\),
  \item \(\hat{r} = \frac{\vec{r}}{r}\) es el vector unitario en la dirección de \(\vec{r}\).
\end{itemize}

\begin{wrapfigure}{r}{0.3\textwidth}
  \centering
  \includegraphics[width=\linewidth]{magnetic_field_of_a_point_charge.png}
  \caption{Campo magnético de una carga puntual.}
  \label{fig:campo_magnético_de_una_carga_puntual}
\end{wrapfigure}
La forma del campo magnético generado por \(q\) es circular, y \(\vec{B}\) es perpendicular a \(\vec{v}\) y \(\vec{r}\). Como se muestra en la figura \ref{fig:campo_magnético_de_una_carga_puntual} el campo magnético \(\vec{B}\) decrece con la distancia al punto \(P\). Además mientras menor es el ángulo entre \(\vec{v}\) y \(\vec{r}\) menor será el campo magnético \(\vec{B}\), y en el caso particular de \(\theta = 0^\circ\) el campo magnético \(\vec{B}\) será cero. 

Este campo magnético es el responsable de que las cargas en movimiento experimenten una fuerza magnética \(\vec{F}_B = q\vec{v} \times \vec{B}\).

Es importante recordar que \textbf{las líneas de campo magnético siempre son cerradas}. Esto lo vimos en la sección \ref{sec:flujo_magnético}, si encerramos un imán en una superficie Gaussiana, todas las líneas de campo magnético que salen, vuelven a entrar. Esto resulta en un flujo nulo.

\begin{tcolorbox}[myconclusion]
  El campo magnético total generado por varias cargas en movimiento es la suma vectorial de los campos generados por las cargas individuales.
\end{tcolorbox}

Con esto en mente, podemos encontrar el campo magnético generado por un diferencial de carga. De la sección \ref{sec:movimiento_de_cargas} sabemos que para un pedacito de segmento de longitud \(dl\) la carga es:
\[
dq = nqA\, dl
\]

De la ecuación \ref{eq:campo_magnético_de_una_carga_puntual} podemos aplicar la definición geométrica del producto vectorial, y reemplazar la carga por la expresión anterior. Entonces el campo magnético \(dB\) generado por un diferencial de carga es:
\begin{align*}
  dB &= \frac{\mu_0}{4\pi} \frac{dq \, v_d \, \sin\theta}{r^2} \\
  dB &= \frac{\mu_0}{4\pi} \frac{\left(nq v_d A\right) dl}{r^2} \sin\theta \\ 
  dB &= \frac{\mu_0}{4\pi} \frac{I dl}{r^2} \sin\theta 
\end{align*}

Entonces el un diferencial de campo magnético \(d\vec{B}\) generado por una corriente \(I\) que fluye a través de un pequeño segmento \(dl\) es:
\begin{equation}
  \boxed{d\vec{B} = \frac{\mu_0}{4\pi} \frac{I \, d\vec{l} \times \hat{r}}{r^2}}
  \label{eq:campo_magnetico_de_un_diferencial_de_corriente}
\end{equation}

\subsubsection{Ley de Biot-Savart}

La fórmula \ref{eq:campo_magnético_de_una_carga_puntual} es una versión simplificada de la Ley de Biot-Savart para una carga puntual. A partir de \eqref{eq:campo_magnetico_de_un_diferencial_de_corriente} podemos obtener la Ley de Biot-Savart que establece que el campo magnético \(\vec{B}\) generado por una corriente \(I\) que fluye a través de una curva \(C\) es:

\begin{equation}
  \vec{B} = \frac{\mu_0}{4\pi} \int_C \frac{I \, \vec{dl} \times \hat{r}}{r^2},
\end{equation}
donde:
\begin{itemize}
  \item \(\mu_0 = 4 \pi \times 10^{-7} \, \frac{\si{\newton}}{\si{\ampere\squared}}\) es la permeabilidad del vacío,
  \item \(\vec{dl}\) es el vector diferencial de longitud de la curva \(C\),
  \item \(r = |\vec{r}|\) es la magnitud del vector \(\vec{r}\),
  \item \(\hat{r} = \frac{\vec{r}}{r}\) es el vector unitario en la dirección de \(\vec{r}\).
\end{itemize}

\subsubsection{Aplicación de la ley de Biot-Savart}
\label{sec:aplicacion_de_la_ley_de_biot_savart}

La ley de Biot-Savart es una ecuación general que describe el campo magnético generado por una corriente. Sin embargo, en la práctica, su aplicación puede ser compleja, especialmente cuando se trata de curvas complejas o cuando se requiere resolver integrales complejas.

En la aplicación práctica de la ley de Biot-Savart, es común simplificar la integración considerando curvas rectas o cilíndricas, lo que permite resolver integrales más simples. Incluso es posible determinar la dirección del campo magnético con la regla de la mano derecha y calcular únicamente el módulo del campo magnético. A continuación se dan los resultados de algunos de los casos más comunes.

Para un segmento largo y recto de cable el módulo del campo magnético se convierte en:
\begin{equation*}
  B = \frac{\mu_0 I}{2\pi r}
\end{equation*}
donde:
\begin{itemize}
  \item \(\mu_0 = 4 \pi \times 10^{-7}\) es la permeabilidad del vacío,
  \item \(I\) es la corriente que fluye a través del cable,
  \item \(r\) es la distancia desde el cable al punto donde se evalúa el campo magnético.
\end{itemize}

\noindent Para una espira circular de radio \(R\) el módulo del campo magnético en el centro es:
\begin{equation*}
  B = \frac{\mu_0 I}{2R}
\end{equation*}

\noindent Para un conjunto de \(N\) espiras de radio \(R\) el campo en el centro es:
\begin{equation*}
  B = \mu_0 \frac{N I}{2R}
\end{equation*}
Este resultado se refiere a \(N\) espiras circulares (o una bobina plana) de radio \(R\), todas concentradas en el mismo plano y con el mismo centro.

\subsubsection{Ley de Ampère}

Hasta el momento, el cálculo del \hl{campo magnético} generado por una corriente ha sido realizado a través de la Ley de Biot-Savart. Sin embargo, esta ley no es la única ecuación que describe el campo magnético generado por una corriente. La Ley de Ampère es una de las ecuaciones fundamentales del electromagnetismo y establece una relación entre el campo magnético \(\vec{B}\) y las corrientes eléctricas que lo generan. 

La Ley de Ampère puede parecer abstracta al principio, pero su lógica es similar a la Ley de Gauss, aunque aplicada al magnetismo. Esta ley establece una relación entre el campo magnético \(\vec{B}\) y las corrientes eléctricas que lo generan. 
\[
\oint_{\text{lazo cerrado}} \vec{B} \cdot d\vec{l} = \mu_0 \, I_{\text{enc}}
\]
donde:
\begin{itemize}
  \item \(\vec{B}\): Campo magnético.
  \item \(d\vec{l}\): Elemento infinitesimal de un camino cerrado (lazo).
  \item \(I_{\text{enc}}\): Corriente neta encerrada por el lazo.
  \item \(\mu_0\): Permeabilidad magnética del vacío (\(4\pi \times 10^{-7} \, \mathrm{T \cdot m/A}\)).
\end{itemize}

\begin{tcolorbox}[myconclusion]
\textbf{¿Sobre qué se integra?}

En la Ley de Ampère, no se integra sobre una superficie, sino sobre un camino cerrado (lazo o bucle). Este camino cerrado se llama lazo amperiano, y es análogo a la superficie gaussiana en la Ley de Gauss, pero con diferencias importantes:
\begin{itemize}
  \item Ley de Gauss: Integral sobre una superficie cerrada.
  \subitem Relaciona el flujo eléctrico (\(\oint \vec{E} \cdot d\vec{A}\)) con la carga encerrada.  
  \item Ley de Ampère: Integral sobre un camino cerrado (una línea).
  \subitem Relaciona la circulación del campo magnético (\(\oint \vec{B} \cdot d\vec{l}\)) con la corriente encerrada.  
\end{itemize}
\end{tcolorbox}

Un camino amperiano es un camino imaginario cerrado que tú defines, al igual que la superficie gaussiana. Debe elegirse estratégicamente para aprovechar la simetría del problema (por ejemplo, círculos concéntricos alrededor de un alambre recto). 

\paragraph{¿Qué representa la integral \(\oint \vec{B} \cdot d\vec{l}\)?}

La integral \(\oint \vec{B} \cdot d\vec{l}\) se llama circulación del campo magnético y representa la suma de las componentes tangenciales de \(\vec{B}\) a lo largo del lazo amperiano. En términos físicos:
\[
\oint \vec{B} \cdot d\vec{l} = \mu_0 \, I_{\text{enc}}
\]
\begin{itemize}
  \item Lado izquierdo: Circulación del campo magnético (¿cómo ``rota'' \(\vec{B}\) alrededor del lazo?).  
  \item Lado derecho: Corriente total de conducción que atraviesa la superficie encerrada por el lazo.  
\end{itemize}

\paragraph{¿Qué buscamos con la Ley de Ampère?}

El objetivo principal es calcular el \textbf{campo magnético} (\(\vec{B}\)) en situaciones con alta simetría, donde la integral \(\oint \vec{B} \cdot d\vec{l}\) se simplifica. Por ejemplo: En un alambre infinito, un solenoide, o un toroide.  

\begin{itemize}
  \item Alambre infinito recto:  
    \[
    B = \frac{\mu_0 I}{2\pi r} \quad \text{(dirección circular alrededor del alambre)}.
    \]
  \item Solenoide ideal:  
     \[
     B = \mu_0 n I \quad \text{(campo uniforme en el interior)}.
     \]
  \item Toroide:  
     \[
     B = \frac{\mu_0 N I}{2\pi r} \quad \text{(campo circular dentro del toroide)}.
     \]
\end{itemize}
Nótese que los resultados son consistentes con la Ley de Biot-Savart. Para el alambre infinito (o largo) y recto se ha obtenido el mismo resultado. Aunque usted tal vez se pregunte ¿Qué paso con el solenoide ideal? Parece que no ha dado lo mismo que el ejemplo del conjunto de espiras. Esto es porque el solenoide tiene la forma de un cilindro, como se vió en la figura \ref{fig:solenoide}. Un solenoide es una bobina alargada con \(N\) espiras distribuidas uniformemente a lo largo de una longitud \(L\). El ejemplo de las espiras en la sección \ref{sec:aplicacion_de_la_ley_de_biot_savart} es un conjunto de espiras que no tiene forma de cilindro, sinó que están todas en el mismo plano.

\paragraph{Ejemplo práctico: Cálculo del campo magnético en un solenoide}

Consideremos un solenoide de \(N\) vueltas, longitud \(L\), y corriente \(I\):

\textbf{Definimos el lazo amperiano}: Rectángulo que abarca parte del solenoide.

\textbf{Aplicación de Ampère}: 
\[
  \oint \vec{B} \cdot d\vec{l} = B \cdot L_{\text{interior}} = \mu_0 n I L_{\text{interior}},
\]
donde \(n = N/L\) (vueltas por unidad de longitud).  

\textbf{Resultado}: 
\[
  B = \mu_0 n I.
\]