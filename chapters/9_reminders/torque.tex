\subsection{Torque y Momento de Inercia}
\label{sec:torque_y_momento_de_inercia}

El torque determina cómo una fuerza causa rotación, mientras que el momento de inercia cuantifica la resistencia a cambios en el movimiento rotacional. Juntos, describen la dinámica angular mediante \( \tau = I\alpha \), similar a \( F = ma \) en movimiento lineal.

\begin{tcolorbox}[remember, title=Torque (\(\tau\))]
  Es la medida de la capacidad de una fuerza para producir rotación alrededor de un eje. Análogo rotacional de la fuerza en movimiento lineal.  
  \[
    \tau = \mathbf{r} \times \mathbf{F} \quad \text{(vectorial)}, \quad \tau = rF\sin\theta \quad \text{(magnitud)}
  \]  
  Donde:
  \begin{itemize}
    \item \( r \): Brazo de palanca (distancia desde el eje hasta el punto de aplicación de la fuerza).
    \item \( F \): Magnitud de la fuerza aplicada.
    \item \( \theta \): Ángulo entre el vector fuerza y el brazo de palanca.
  \end{itemize}
  \textbf{Unidad:} Newton-metro (N·m).  

  \noindent\textbf{Factores clave:} Depende de la magnitud de la fuerza, su dirección y el punto de aplicación. El torque máximo se obtiene cuando la fuerza es perpendicular al brazo (\( \theta = 90^\circ \)).
\end{tcolorbox}

\begin{tcolorbox}[remember, title=Momento de Inercia (\(I\))]
  Medida de la resistencia de un cuerpo a la aceleración angular. Análogo rotacional de la masa en movimiento lineal.  
  \[
    I = \sum m_i r_i^2 \quad \text{(sistema de partículas)}, \quad I = \int r^2 \, dm \quad \text{(cuerpo continuo)}
  \]  
  \textbf{Dependencia:} Aumenta con la distancia de la masa al eje de rotación. Ejemplos comunes:
  \begin{itemize}
    \item Esfera sólida: \( I = \frac{2}{5} mR^2 \).
    \item Varilla delgada (eje en centro): \( I = \frac{1}{12} mL^2 \).
    \item Disco sólido: \( I = \frac{1}{2} mR^2 \).
  \end{itemize}
  \textbf{Unidad:} \(kg\cdot m^2\).  
\end{tcolorbox}

\begin{tcolorbox}[remember, title=Relación Fundamental (Segunda Ley de Newton para Rotación)]
  Esta relación es como \(F=ma\) pero para las rotaciones. \(I\) equivale a la masa y \(\alpha\) a la aceleración.
  \[
  \tau_{\text{neto}} = I \alpha
  \]  
  Donde \( \alpha \) es la aceleración angular. Esto implica que, para un torque dado, un mayor momento de inercia resulta en menor aceleración angular.
\end{tcolorbox}