Resumen final de ecuaciones clave

- Ley de Ohm local:  
\[
\vec{J} = \sigma \vec{E}
\]

- Ley de Ohm macroscópica:  
\[
I = \frac{\sigma A}{L} \Delta V
\]

- Definición de resistencia:  
\[
R = \frac{\Delta V}{I} = \rho \frac{L}{A}
\]

- Relación entre corriente y densidad de corriente:  
\[
I = \int_A \vec{J} \cdot d\vec{A}
\]

Estas ecuaciones permiten modelar y analizar cualquier sistema de corriente continua en conductores ohmicos.



Resumen final de fórmulas clave

Serie:
- Corriente común: \( I = \text{constante} \)
- Tensión total: \( V = V_1 + V_2 + \cdots + V_n \)
- Resistencia equivalente:  
\[
\boxed{R_{\text{eq}} = \sum_{i=1}^{n} R_i}
\]

Paralelo:
- Tensión común: \( V = \text{constante} \)
- Corriente total: \( I = I_1 + I_2 + \cdots + I_n \)
- Resistencia equivalente:  
\[
\boxed{\frac{1}{R_{\text{eq}}} = \sum_{i=1}^{n} \frac{1}{R_i}}
\]

