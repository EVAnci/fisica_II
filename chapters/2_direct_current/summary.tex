Resumen final de ecuaciones clave

\begin{tcolorbox}[title=Corriente Eléctrica]
  La corriente eléctrica es la tasa a la cual circula la carga a traves de una superficie.
  \[
    I_{prom} = \frac{\Delta Q}{\Delta t}
  \]
  que puede relacionarse con el movimiento de las cargas:
  \[
    I_{prom} = n q v_d A
  \]

  De esta ecuación se puede obtener la velocidad a la cual circulan los portadores de carga. La velocidad de deriva o arrastre es:
  \[
    v_d = \frac{I}{nqA}
  \]
  donde:
  \begin{itemize}
    \item \(I\) es la corriente que circula por el material
    \item \(q\) es la carga de los portadores (\(e=1.6\times10^{-19}\) en el caso de los electrones)
    \item \(A\) es el área transversal del material
    \item \(n\) es la densidad de portadores (el número de portadores) del el material por unidad de volumen.
  \end{itemize}

  Para obtener \(n\) para un material dado podemos usar la siguiente aproximación:
  \[
    n=\frac{\delta}{m_{molar}} N_A
  \]
  donde:
  \begin{itemize}
    \item \(\delta\) es la densidad del material
    \item \(m_{molar}\) es la masa molar del material
    \item \(N_A = 6.022 \times 10^{23} \, \mathrm{mol}^{-1}\) es el número de Avogadro.
  \end{itemize}
\end{tcolorbox}

\begin{tcolorbox}[title=Ley de Ohm]
  La ley de Ohm establece una relación de proporcionalidad entre la corriente, el potencial y la resistencia:
  \[
    V = IR
  \]
  \begin{itemize}
    \item La densidad de corriente es: \(\vec{J} = \sigma \vec{E}\)
    \item Ley de Ohm macroscópica: \(I = \frac{\sigma A}{L} \Delta V\)
    \item Definición de resistencia: \(R = \frac{\Delta V}{I} = \rho \frac{L}{A}\)
    \item Relación entre corriente y densidad de corriente: \(I = \int_A \vec{J} \cdot d\vec{A}\)
  \end{itemize}
  Estas ecuaciones permiten modelar y analizar cualquier sistema de corriente continua en conductores ohmicos.
\end{tcolorbox}


\begin{tcolorbox}[title=Circuitos]
  \paragraph{Serie:}
  \begin{itemize}
    \item Corriente común: \( I = \text{constante} \)
    \item Tensión total: \( V = V_1 + V_2 + \cdots + V_n \)
    \item Resistencia equivalente: \(R_{\text{eq}} = \sum_{i=1}^{n} R_i\)
  \end{itemize}
  
  \paragraph{Paralelo:}
  \begin{itemize}
    \item Tensión común: \( V = \text{constante} \)
    \item Corriente total: \( I = I_1 + I_2 + \cdots + I_n \)
    \item Resistencia equivalente: \(\frac{1}{R_{\text{eq}}} = \sum_{i=1}^{n} \frac{1}{R_i}\)
  \end{itemize}
\end{tcolorbox}


\begin{tcolorbox}[title=Trabajo y Potencia eléctrica]
  \paragraph{Potencia general:}  
  
  \[
    P = V I
  \]
  
  \paragraph{Potencia disipada en resistencias:}  
  \[
    P = I^2 R = \frac{V^2}{R}
  \]
  
  \paragraph{Energía disipada o entregada en el tiempo \( \Delta t \):}
  \[
    W = P \Delta t = V I \Delta t = I^2 R \Delta t = \frac{V^2}{R} \Delta t
  \]
\end{tcolorbox}
