\subsection{Resumen}

\begin{tcolorbox}[remember, title=Imágenes]
  \subparagraph{Imágenes reales vs. imágenes virtuales}
  \textbf{Imagen real}:
  \begin{itemize}
    \item Se forma por la \textbf{intersección real} de rayos reflejados.
    \item Puede proyectarse en una pantalla.
    \item Se forma \textbf{frente al espejo} (en espejos cóncavos, si el objeto está a cierta distancia).
    \item Es \textbf{invertida} respecto al objeto.
  \end{itemize}

  \textbf{Imagen virtual}:
  \begin{itemize}
    \item Se forma por la \textbf{prolongación} de rayos reflejados, que no se cruzan realmente.
    \item No se puede proyectar en una pantalla.
    \item Se forma \textbf{detrás del espejo}.
    \item Es \textbf{derecha} (no invertida) respecto al objeto.
  \end{itemize}

  \subparagraph{Imágenes aumentadas, disminuidas o del mismo tamaño}

  Depende de la \textbf{posición del objeto} respecto al espejo y de la \textbf{curvatura del espejo}:

  En un espejo \textbf{plano}, la imagen es \textbf{virtual}, \textbf{derecha}, y del \textbf{mismo tamaño} que el objeto.
  
  En un espejo \textbf{cóncavo}:
  \begin{itemize}
    \item Si el objeto está \textbf{lejos del foco}, se forma una imagen \textbf{real, invertida y más pequeña o del mismo tamaño}.
    \item Si está \textbf{entre el foco y el espejo}, se forma una imagen \textbf{virtual, derecha y aumentada}.
  \end{itemize}
  
  En un espejo \textbf{convexo}, siempre se forma una imagen \textbf{virtual, derecha y disminuida}.
  
\end{tcolorbox}