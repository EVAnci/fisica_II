\section*{Demostraciones Adicionales}

\subsection*{Teorema de la suma de los ángulos internos de un triángulo}
\label{sec:suma_angulos_internos}

Para demostrar que la suma de dos ángulos internos de un triángulo es igual al ángulo externo no adyacente, se puede usar una demostración geométrica sencilla basada en las propiedades de los ángulos formados por líneas paralelas y la suma de los ángulos interiores de un triángulo.

\paragraph{Teorema a demostrar}

En cualquier triángulo, el ángulo exterior es igual a la suma de los dos ángulos interiores no adyacentes.

\paragraph{Demostración}

Consideremos un triángulo \(\triangle ABC\), donde el lado \(BC\) se prolonga hacia un punto \(D\), formando un ángulo exterior en el vértice \(C\), al que llamaremos \(\angle ACD\). Los ángulos interiores del triángulo son \(\angle A\), \(\angle B\), y \(\angle C\).

Queremos demostrar que:

\[
\angle A + \angle B = \angle ACD
\]

\subparagraph{Paso 1: Suma de ángulos interiores de un triángulo}

Sabemos que la suma de los ángulos interiores de un triángulo es 180°:

\[
\angle A + \angle B + \angle C = 180^\circ
\]

\subparagraph{Paso 2: Ángulo suplementario}

El ángulo exterior \(\angle ACD\) y el ángulo interior adyacente \(\angle C\) forman un par lineal, es decir, están sobre una línea recta. Por lo tanto:

\[
\angle C + \angle ACD = 180^\circ
\]

\subparagraph{Paso 3: Sustitución}

Igualando las ecuaciones (1) y (2):

\[
\angle A + \angle B + \angle C = \angle C + \angle ACD
\]

Restamos \(\angle C\) de ambos lados:

\[
\angle A + \angle B = \angle ACD
\]
