\section*{Prefacio}

Este libro fue escrito para comprender tanto conceptualmente como analíticamente, tal y como se requiere en un curso de física, los temas dictados en un curso de física dos. En general, los temas que se abordan incluyen 
\begin{itemize}
  \item Electromagnetísmo
  \item Ondas y vibraciones
  \item Luz y ondas electromagnéticas 
  \item Estática y dinámica de fluidos
  \item Termodinámica
\end{itemize}

Un importante detalle sobre este libro, es que ha sido escrito desde cero en \LaTeX, y todo el código fuente está almacenado en un repositorio en GitHub. Si usted desea aportar sus conocimientos y escribir parte del mismo, puede enviar un correo electrónico a \texttt{e.anci@alumno.um.edu.ar}.

La idea principal de la creación de este documento reside en la libertad del conocimiento. Tal y como ocurre con el software libre y el privativo, la propiedad intelectual priva a personas de poder estudiar.  Este documento es una fuente de conocimiento que busca ser rigurosa y detallada, tal y como los mejores libros de física didáctica, pero con la diferencia de dar total libertad al lector. Este documento puede descargarlo, publicarlo, modificarlo y utilizarlo libremente. Para más información sobre cómo puede usarlo y qué puede hacer visite \url{https://creativecommons.org/licenses/by-sa/4.0/} (Licencia CC-BY-SA 4.0).

\section*{¿Cómo aportar?}

Si desea aportar contenido al libro debe cumplir una serie de requisitos 
\begin{itemize}
  \item Estar dispuesto a aprender.
  \item Tener un conocimiento mínimo sobre \LaTeX.
  \item Saber usar de forma básica \texttt{git}.
  \item Por obvias razones, tener conocimientos de física.
\end{itemize}

Si alguno de estos requisitos no los satisface, puede seguir alguna guía sencilla o intentar aprender sobre la marcha. Aunque estos requisitos parezcan un capricho personal, realmente es lo mínimo que uno debe saber para poder:
\begin{enumerate}
  \item trabajar en equipo de forma distribuida (\texttt{git}),
  \item lograr una calidad tipográfica de alta calidad (\LaTeX),
  \item y escribir sobre física.
\end{enumerate}

Si satisface estos requisitos puede descargar el código fuente en \url{https://github.com/EVAnci/fisica_ii}. Todas las instrucciones de compilación y detalles necesarios están en un archivo llamado \texttt{README.md}.

\begin{center}
  \ccbysa
\end{center}

\clearpage

\section*{Reporte de errores}

Si al leer el documento se encuentra con cualquier tipo de error, puede crear un \texttt{Issue} de GitHub. Estos son muy sencillos de crear, si no sabe como crear uno, puede leer más sobre ello en \href{https://docs.github.com/en/issues/tracking-your-work-with-issues/learning-about-issues/quickstart}{docs.github.com}.

Los errores que reporte deben seguir una serie de requisitos para que sean revisados. Primero debe disponer de la última versión del documento: \href{https://github.com/EVAnci/fisica_II/releases}{descargar aquí}.

Una vez esté seguro de tener la última versión, redacte el error siguiendo esta sugerencia de estructura:
  
\texttt{Título: Error <tipo de error>}, donde el tipo de error debe ser, según su criterio, un título que describa su naturaleza, es decir: ortográfico, conceptual, etcétera.

\texttt{Cuerpo: Capítulo\#-Sección\#-página.}, luego puede colocar la descripción o detalles del error. De esa forma el escritor podrá identificar rápidamente donde está el error y corregirlo.

Cualquier \texttt{Issue} que no describa de forma concreta el problema será cerrado y no se revisará. 

Recuerde que la estructura propuesta en el apartado superior es una sugerencia. Si usted desea usar otra forma de describir el problema que pueda ser entendida por el escritor, es libre de reportar un error siguiendo su propia estructura.
