\section*{Prefacio}

Este libro fue escrito para comprender tanto conceptualmente como analíticamente, los temas dictados en un curso de física dos. En general, los temas que se abordan incluyen 
\begin{itemize}
  \item Electromagnetísmo
  \item Ondas y vibraciones
  \item Luz y ondas electromagnéticas 
  \item Estática y dinámica de fluidos
  \item Termodinámica
\end{itemize}

Un importante detalle sobre este libro, es que ha sido escrito desde cero en \LaTeX, y todo el código fuente está almacenado en un repositorio en GitHub. Si usted desea aportar sus conocimientos y escribir parte del mismo, puede enviar un correo electrónico a \texttt{e.anci@alumno.um.edu.ar}.

La idea principal de la creación de este documento reside en la libertad del conocimiento. Tal y como ocurre con el software libre y el privativo, la propiedad intelectual priva a personas de poder estudiar. Este documento es una fuente de conocimiento que busca ser rigurosa y detallada, tal y como los mejores libros de física didáctica, pero con la diferencia de dar total libertad al lector de poder usar el contenido como desee. Este documento puede descargarlo, publicarlo, modificarlo y utilizarlo libremente respetando la licencia. Para más información sobre cómo puede usarlo y qué puede hacer visite \url{https://creativecommons.org/licenses/by-sa/4.0/} (Licencia CC-BY-SA 4.0).

\section*{¿Cómo aportar o reportar?}

Si desea aportar o reportar sobre contenido del libro, refiera al capítulo \ref{chpt_adicional} para más información.

\begin{center}
  \ccbysa
\end{center}

\clearpage

