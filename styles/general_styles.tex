\usepackage{xcolor}
\usepackage{titlesec}

% Definir colores personalizados
\definecolor{dark_blue}{RGB}{0, 102, 204}
\definecolor{dark_red}{RGB}{204, 0, 0}
\definecolor{soft_green}{RGB}{0, 153, 0}
\definecolor{intense_blue}{RGB}{30, 144, 255}
\definecolor{highlight}{RGB}{255, 230, 153}
\definecolor{lightyellow}{RGB}{255, 249, 171}
\definecolor{lightdanger}{RGB}{255, 205, 210}
% https://latexcolor.com/
\definecolor{brandeisblue}{rgb}{0.0, 0.44, 1.0}
\definecolor{asparagus}{rgb}{0.53, 0.66, 0.42}


\sethlcolor{highlight}
% Cambiar el color de los títulos
\titleformat{\section}
  {\color{dark_blue}\normalfont\Large\bfseries} % Estilo del título de sección
  {\thesection}{1em}{} % Numeración y espacio

\titleformat{\subsection}
  {\color{dark_red}\normalfont\large\bfseries} % Estilo del título de subsección
  {\thesubsection}{1em}{}

\titleformat{\subsubsection}
  {\color{soft_green}\normalfont\normalsize\bfseries} % Estilo del título de subsubsección
  {\thesubsubsection}{1em}{}

\renewcommand{\paragraph}[1]{%
  \noindent\textcolor{orange!70!black}{\textbf{#1}}%
  \hspace{1.2em}%
}

\renewcommand{\subparagraph}[1]{%
  \noindent{\textbf{#1}}%
  \hspace{1.2em}%
}

% Configuración de las listas
\setlist{noitemsep} % Eliminar espacio entre elementos de las listas
\setlist[itemize]{label=\textbullet} % Cambiar el símbolo de las listas itemize


% Definir comandos personalizados
\newcommand{\vect}[1]{\mathbf{#1}} % Vectores en negrita
\newcommand{\diff}[2]{\frac{d#1}{d#2}} % Derivada
\newcommand{\pdiff}[2]{\frac{\partial #1}{\partial #2}} % Derivada parcial

% Sección resaltada para escribir conclusiones bonitas
\tcbset{
  myconclusion/.style={
    enhanced,
    colback=lightyellow,
    colframe=white, % blanco hace invisible el borde
    boxrule=0pt,    % grosor del borde: 0pt = sin borde
    arc=6pt,        % redondeado de bordes
    boxsep=5pt,     % separación entre texto y borde
    left=5pt,
    right=5pt,
    top=5pt,
    bottom=5pt
  }
}

% Sección resaltada para escribir conclusiones bonitas
\tcbset{
  mydanger/.style={
    enhanced,
    colback=lightdanger,
    colframe=white, % blanco hace invisible el borde
    boxrule=0pt,    % grosor del borde: 0pt = sin borde
    arc=6pt,        % redondeado de bordes
    boxsep=5pt,     % separación entre texto y borde
    left=5pt,
    right=5pt,
    top=5pt,
    bottom=5pt
  }
}

\hypersetup{
    colorlinks=true,
    linkcolor=black,
    urlcolor=intense_blue,
    citecolor=black
}