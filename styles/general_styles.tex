\usepackage{xcolor}
\usepackage{titlesec}

% Definir colores personalizados
\definecolor{dark_blue}{RGB}{0, 102, 204}
\definecolor{dark_red}{RGB}{204, 0, 0}
\definecolor{soft_green}{RGB}{0, 153, 0}
\definecolor{intense_blue}{RGB}{30, 144, 255}
\definecolor{highlight}{RGB}{255, 230, 153}
% https://latexcolor.com/
\definecolor{brandeisblue}{rgb}{0.0, 0.44, 1.0}
\definecolor{asparagus}{rgb}{0.53, 0.66, 0.42}

% Cambiar el color de los títulos
\titleformat{\section}
  {\color{dark_blue}\normalfont\Large\bfseries} % Estilo del título de sección
  {\thesection}{1em}{} % Numeración y espacio

\titleformat{\subsection}
  {\color{dark_red}\normalfont\large\bfseries} % Estilo del título de subsección
  {\thesubsection}{1em}{}

\titleformat{\subsubsection}
  {\color{soft_green}\normalfont\normalsize\bfseries} % Estilo del título de subsubsección
  {\thesubsubsection}{1em}{}


% Definir comandos personalizados
\newcommand{\vect}[1]{\mathbf{#1}} % Vectores en negrita
\newcommand{\diff}[2]{\frac{d#1}{d#2}} % Derivada
\newcommand{\pdiff}[2]{\frac{\partial #1}{\partial #2}} % Derivada parcial


\hypersetup{
    colorlinks=true,
    linkcolor=black,
    urlcolor=intense_blue,
    citecolor=black
}